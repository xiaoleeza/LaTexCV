%%%%%%%%%%%%%%%%%%%%%%%%%%%%%%%%%
%                               %
%  LaTex CV                     %
%  Juri Lukkarila               %
%  2018                         %
%                               %
%%%%%%%%%%%%%%%%%%%%%%%%%%%%%%%%%

\documentclass[9pt,a4paper,oneside]{article}

\usepackage[top=1.32cm, bottom=1.45cm, left=1.18cm, right=1.18cm]{geometry}
\usepackage[OT1]{fontenc}           % 8-bit font encoding 
\usepackage[utf8]{inputenc}         % Unicode input encoding
\usepackage[english]{babel}         % spelling
\usepackage{graphicx}               % external graphics
\usepackage{microtype}              % Microtype makes the text really neat
\usepackage[dvipsnames]{xcolor}     % more colors
\usepackage{units}                  % nice fractions in text
\usepackage[full]{textcomp}         % text symbol support
\usepackage[dmyyyy]{datetime}       % date formatting
\usepackage{multirow}               % table multirow columns 
\usepackage{multicol}               % multicolumn pages
\usepackage[inline]{enumitem}       % extended list structures
\usepackage{tabularx}               % table sizing options
\usepackage{hyperref}               % hyperlinks and urls
\usepackage{fancyhdr}               % headers and footers
\usepackage{tikz}                   % tikz graphics
\usepackage{verbatim}               % comment sections
\usepackage[skins]{tcolorbox}       % colorboxes
\usepackage{pgffor}                 % for loops
\usepackage{metalogo}               % latex logo adjustment
\usepackage{ragged2e}               % override ragged commands locally
\usepackage{verbatim}               % comment sections

% Finnish date format
\renewcommand{\dateseparator}{.}    

% fonts
\usepackage{fontspec}               % requires XeLatex
\setmainfont                    [Ligatures=TeX]{Myriad Pro SemiCondensed}
\newfontfamily\titlefont        [Ligatures=TeX]{Myriad Pro Black SemiExtended}
\newfontfamily\namefont         [Ligatures=TeX]{Myriad Pro Black SemiCondensed}
\newfontfamily\titledatefont    [Ligatures=TeX]{Myriad Pro Bold SemiCondensed}
\newfontfamily\workplacefont    [Ligatures=TeX]{Myriad Pro Semibold SemiCondensed}
\newfontfamily\footerfont       [Ligatures=TeX]{Myriad Pro Light SemiCondensed}

% create line below text
\newcommand{\myrule}{\vspace{0.5mm} {\color{lightgray}{\hrule height 0.5pt width \textwidth depth 0pt}} \vspace{1mm}} 

% title formatting
%\newcommand{\titledate}[2]{{\bfseries \fontsize{11.5}{11.5}\selectfont \fadingtext{left color=headcolor, right color=textcolor}{#1}} \hfill \textbf{#2} \myrule} 
\newcommand{\titledate}[2]{{\titledatefont {\fontsize{11}{11}\selectfont \color{textcolor} #1} \hfill {\fontsize{10}{10}\selectfont #2} \myrule}}

% workplace formatting
\newcommand{\workplace}[1]{{\workplacefont #1}}

% section title formatting
\newcommand{\sectitle}[1]{{\Large \namefont #1}}

% repeat command
\newcommand{\myrepeat}[2]{\foreach \n in {1,...,#2}{#1}}

% draw colored circles
\newcommand{  \bigcircle}[1]{\raisebox{-0.15\baselineskip}{\tikz\draw[#1,fill=#1] (0,0) circle (5.2pt);}}   
\newcommand{\smallcircle}[1]{\raisebox{-0.10\baselineskip}{\tikz\draw[#1,fill=#1] (0,0) circle (4.2pt);}}   

\newcommand{\bcFive} {\bigcircle{circcolor} \bigcircle{circcolor} \bigcircle{circcolor} \bigcircle{circcolor} \bigcircle{circcolor}}
\newcommand{\bcFour} {\bigcircle{circcolor} \bigcircle{circcolor} \bigcircle{circcolor} \bigcircle{circcolor} \bigcircle{lightgray}}
\newcommand{\bcThree}{\bigcircle{circcolor} \bigcircle{circcolor} \bigcircle{circcolor} \bigcircle{lightgray} \bigcircle{lightgray}}
\newcommand{\bcTwo}  {\bigcircle{circcolor} \bigcircle{circcolor} \bigcircle{lightgray} \bigcircle{lightgray} \bigcircle{lightgray}}
\newcommand{\bcOne}  {\bigcircle{circcolor} \bigcircle{lightgray} \bigcircle{lightgray} \bigcircle{lightgray} \bigcircle{lightgray}}

\newcommand{\scFive} {\smallcircle{circcolor} \smallcircle{circcolor} \smallcircle{circcolor} \smallcircle{circcolor} \smallcircle{circcolor}}
\newcommand{\scFour} {\smallcircle{circcolor} \smallcircle{circcolor} \smallcircle{circcolor} \smallcircle{circcolor} \smallcircle{lightgray}}
\newcommand{\scThree}{\smallcircle{circcolor} \smallcircle{circcolor} \smallcircle{circcolor} \smallcircle{lightgray} \smallcircle{lightgray}}
\newcommand{\scTwo}  {\smallcircle{circcolor} \smallcircle{circcolor} \smallcircle{lightgray} \smallcircle{lightgray} \smallcircle{lightgray}}
\newcommand{\scOne}  {\smallcircle{circcolor} \smallcircle{lightgray} \smallcircle{lightgray} \smallcircle{lightgray} \smallcircle{lightgray}}

% custom list
\newlist{cvlist}{itemize}{1}
\setlist[cvlist]{leftmargin=0.32cm, itemsep=0.5mm,parsep=0pt,topsep=0.5mm,label={\raisebox{-0.06\baselineskip}{\Large \textbullet}},labelsep = 1mm, before=\small, font=\normalfont}

% colors
\definecolor{headcolor}  {cmyk}{   1, 0.95, 0,    0} % CV
\definecolor{textcolor}  {cmyk}{   1, 0.80, 0,    0} % title text
\definecolor{refcolor}   {cmyk}{   1, 0.75, 0,    0} % urls
\definecolor{circcolor}  {cmyk}{   1, 0.70, 0,    0} % color of circles etc
\definecolor{boxcolor}   {cmyk}{0.03, 0.03, 0, 0.08} % box frame
\definecolor{backcolor}  {cmyk}{   0,    0, 0, 0.01} % box background
\definecolor{lightgray}  {cmyk}{0.05, 0.05, 0, 0.15} % circle secondary color

% Path of graphics files. By default root and figures directory are used
\graphicspath{{./}{figures/}}

% Footer
\pagestyle{fancy}
\fancyhead{}
\fancyfoot[L]{ \footnotesize \footerfont CV}
\fancyfoot[C]{ \footnotesize \footerfont Juri Akseli Lukkarila}
\fancyfoot[R]{ \footnotesize \footerfont \today}

\renewcommand{\headrulewidth}{0pt}
\renewcommand{\footrulewidth}{0.8pt}

\setlength{\footskip}{22pt}             % footer spacing

% paragraph breaks
\setlength{\parindent}{0pt}         % paragraph indentation
\setlength{\parskip}{4pt plus 1pt minus 2pt} % paragraph lineskip

% some extra vertical spacing
\renewcommand{\baselinestretch}{1.01}

% linewidth for vertical line separating the two main columns
\setlength{\arrayrulewidth}{0.8pt} 

% relax page height equalization
\raggedbottom

% Length variables
\newcommand{\boxarc}{2.25mm}
\newcommand{\boxtop}{1.5mm}
\newcommand{\boxmargins}{1mm}
\newcommand{\mysep}{\hspace{2.8mm}}
\newcommand{\boxskip}{\vspace{0.9\baselineskip}}

% tcolorbox settings
\tcbset{colback=backcolor,colframe=boxcolor,colbacktitle=boxcolor,coltitle=black, segmentation style={textcolor}}           

% adjust LaTex logo kerning
\setlogokern{La}{-.18em}
\setlogokern{aT}{-.075em}

% hyperref settings and pdf metadata
\hypersetup{pdfpagemode=UseNone,
    colorlinks=true,urlcolor=refcolor,linkcolor=black,citecolor=black,
    pdftitle={CV Juri Lukkarila},
    pdfauthor={Juri Lukkarila},
    pdfsubject={CV}}

% picture with round frame
\newcommand{\roundpic}[4][]{
    \tikz\node [circle, minimum width = #2, path picture = {
        \node [#1] at (path picture bounding box.center) {
            \includegraphics[width=#3]{#4}};}] {};
}

% logo with text
\newcommand*{\logopic}[1]{
    \raisebox{-0.35\baselineskip}{ % negative raisebox lowers the logo
        \includegraphics[
        height=1.25\baselineskip, % logo size
        width=1.25\baselineskip,
        keepaspectratio,
        ]{#1}}
}

% color fade
\usetikzlibrary{fadings}
\newcommand\fadingtext[3][]{%
    \begin{tikzfadingfrompicture}[name=fading letter]
        \node[text=transparent!0,inner xsep=0pt,outer xsep=0pt,#1] {#3};
    \end{tikzfadingfrompicture}%
    \begin{tikzpicture}[baseline=(textnode.base)]
    \node[inner sep=0pt,outer sep=0pt,#1](textnode){\phantom{#3}}; 
    \shade[path fading=fading letter,#2,fit fading=false]
    (textnode.south west) rectangle (textnode.north east);% 
    \end{tikzpicture}% 
}

%%%%%%%%%%%%%%%%%%%%%%%%%%%%%%%%%%%%%%%%%%%%%%%%%%%%%%%%%%%%%%%%%%%%%%%%%%%%%%
%%%%%%%%%%%%%%%%%%%%%%%%%%%%%%%%%% DOCUMENT %%%%%%%%%%%%%%%%%%%%%%%%%%%%%%%%%%

\begin{document}
\setlength{\fboxsep}{0pt}
% TITLE TEXT
\begin{minipage}{0.72\textwidth}
    % CV
    \begin{minipage}{0.25\textwidth}
        \vspace{-8mm}
        % fading text not showing up in some web pdf readers :(
        %{\fontsize{60}{60}\selectfont \textbf{\fadingtext[scale=1]{left color=headcolor,middle color = refcolor, right color=refcolor}{CV}}} \vspace{1mm}
        {\titlefont \fontsize{60}{60}\selectfont \color{headcolor}{C}\color{textcolor}{V}} \vspace{1mm}
    \end{minipage}
    % NAME
    \begin{minipage}{0.75\textwidth}
        \vspace{-8mm}
        {\huge \namefont Juri Akseli Lukkarila \vspace{0.5mm}} \\
        Born 14.5.1989, Finland \\
        Malminkatu 24 A 3, 00100 Helsinki \vspace{1mm}
    \end{minipage}
    \vspace{0.8mm}
    \rule{0.99\textwidth}{1.2pt} %\hspace{1mm}
    \small
    \setlength{\columnsep}{0.5cm}
    \setlength{\multicolsep}{0cm}
    % CONTACT INFO WITH LOGOS
    \begin{multicols}{3}
        \begin{itemize}[label={},leftmargin=0.4cm,labelwidth=0cm, labelsep=0cm, itemsep=-0.25mm]
            \item \logopic{call.pdf} \ \href{tel:+358400641594}{+358 400 641 594}
            \item \logopic{github.pdf} \ \href{https://github.com/Esgrove}{Esgrove}
            \item \logopic{mail.pdf} \ \href{mailto:esgrove@outlook.com}{esgrove@outlook.com}
            \item \logopic{In-2C-128px-R.png} \ \href{https://www.linkedin.com/in/jlukkari/}{linkedin.com/in/jlukkari}
            \item \logopic{web.pdf} \ \href{www.esgrove.fi}{esgrove.fi}
            \item \logopic{twitter.pdf} \ \href{https://twitter.com/djesgrove}{djesgrove} 
        \end{itemize}
    \end{multicols}
\end{minipage}
% PHOTO
\begin{minipage}{0.28\textwidth}
    \raggedleft
    \vspace{-5mm}
    \roundpic[xshift=-3.70cm,yshift=-0.1cm]{3.5cm}{4.3cm}{Jook Joint.png}
\end{minipage}
    
\vspace{2mm} % space after title section
    
\begin{minipage}[t]{0.638\textwidth}
    
    %%%%%%%% EDUCATION %%%%%%%%
    
    \begin{tcolorbox}[top=1mm,bottom=1mm,right=1mm,left=1mm,arc=\boxarc,toptitle=\boxtop,bottomtitle=1mm,title=\sectitle{EDUCATION},box align=top]
        \titledate{Master of Science (Technology)}{12/2017}
        \workplace{Aalto University School of Electrical Engineering} \\
        Master's Programme in Computer, Communication and Information Sciences \\
        \hspace*{0.25mm} {\Large \textbullet} Major: \textit{Acoustics and Audio Technology} \\
        
        \titledate{Bachelor of Science (Technology)}{02/2016}
        \workplace{Aalto University School of Electrical Engineering} \\
        Degree Programme in Electronics and Electrical Engineering \\
         \hspace*{0.25mm} {\Large \textbullet} Major:  \textit{Electronics} \hspace{3mm} {\Large \textbullet} Minor:  \textit{Computer Science} \\
        
        \titledate{Finnish Matriculation Examination}{05/2008}
        \workplace{Oulu University Upper Secondary School}
    \end{tcolorbox} \boxskip
    
    %%%%%%%% EXPERIENCE %%%%%%%%
    \begin{tcolorbox}[top=1mm,bottom=1mm,right=1mm,left=1mm,arc=\boxarc,toptitle=\boxtop,bottomtitle=1mm,title=\sectitle{EXPERIENCE},box align=bottom]
        
        \titledate{Project Employee}{12/2017 -- 09/2018}
        \workplace{Aalto University Department of Signal Processing and Acoustics} \\
        Research and development work in the Aalto speech recognition research group:
        \begin{cvlist}
            \item Building and administering a server-based, real-time automatic speech recognition system (ASR) running on a RHEL 7 virtual server, enabling flexible speech audio input streaming.
            \item Visiting Researcher at the Department of Computer Science at the University of Sheffield, UK, during 4.6.-6.7.2018, focusing on audio beamforming and microphone arrays for ASR.
            \item Advisor for a Master's Thesis on developing an augmented reality automatic speech recognition application on iOS by utilizing the Apple ARKit with our ASR server. \\
        \end{cvlist}
        
        \titledate{Research Assistant}{12/2016 -- 10/2017}
        \workplace{Aalto University Department of Signal Processing and Acoustics} \\
        I did my Master’s Thesis at the Aalto speech recognition research group, working in an Academy of Finland project on developing a conversational assistance application for the hearing impaired by utilizing automatic speech recognition.
        \begin{cvlist}
            \item Developed a GUI desktop application for Finnish real-time speech recognition running on Ubuntu using Python, GTK+, GStreamer and the Kaldi automatic speech recognition toolkit.
            \item Designed and conducted user testing to validate the developed software application. \\
        \end{cvlist}
        
        \titledate{DJ \& Radio Host}{04/2011 -- present}
        \workplace{Basso Media} \\
        I co-host a biweekly two hour radio show \textit{Ruff Cut} on Finland’s foremost urban and electronic music radio station \textit{Bassoradio} every other Thursday evening. \\
        
        \titledate{DJ \& Audio Engineer}{01/2010 -- present}
        \workplace{T:mi Juri Lukkarila} \\
        Accomplished professional DJ with well over 400 performances all around Finland. I also do audio recording, mixing and production as a freelancer. Past projects include radio commercials, voice-overs, theater productions and numerous dance shows.
        \begin{cvlist}
            \item Learned to be self-imposed and to take responsibility of my work and time management.
            \item Has taught me excellent social, communication and organizational skills. \\
        \end{cvlist}
        
        \titledate{Salesperson }{02/2010 -- 01/2013}
        \workplace{Stockmann Herkku Helsinki} \\
        Part-time salesperson at the Stockmann flagship grocery store.
        \begin{cvlist}
            \item Customer service and inventory management in a fast-paced working environment.
        \end{cvlist}
    \end{tcolorbox}
    
\end{minipage}%} 
\hfill\vline\hfill
%\fbox{
\begin{minipage}[t]{0.328\textwidth}
    
\renewcommand{\sp}{\hspace{0mm}}
    
%%%%%%%% SKILLS %%%%%%%%
\begin{tcolorbox}[top=1mm,bottom=1mm,right=1mm,left=1mm,arc=\boxarc,toptitle=\boxtop,bottomtitle=1mm,title=\sectitle{SKILLS},box align=top]
    {\linespread{1.08}\selectfont
    Windows             \hfill \bcFive \\
    MacOS               \hfill \bcFive \\
    Linux               \hfill \bcFour \\

    Office              \hfill \bcFive \\
    MATLAB              \hfill \bcFive \\
    RStudio             \hfill \bcTwo \\
    Visual Studio       \hfill \bcThree \\
    Adobe Illustrator   \hfill \bcFour \\
    Ableton Live        \hfill \bcFive \\
    FL Studio           \hfill \bcFour \\

    Python              \hfill \bcFive \\
    C++                 \hfill \bcFour \\
    C                   \hfill \bcThree \\
    Max/MSP             \hfill \bcTwo \\
    \LaTeX              \hfill \bcFive \par} \vspace{\baselineskip}
    
    Frameworks, libraries and tools I have used (in descending order of familiarity): \\
    {\small \textit{Qt5, OpenMP, Git, OpenGL, CUDA, SFML, GTK+, NumPy, JUCE, Wwise, Boost, UE4, Docker, Unity}}
\end{tcolorbox} \boxskip

%%%%%%%% LANGUAGES %%%%%%%%
\begin{tcolorbox}[top=1mm,bottom=1mm,right=1mm,left=1mm,arc=\boxarc,toptitle=\boxtop,bottomtitle=1mm,title=\sectitle{LANGUAGES},box align=top]
    {\linespread{1.08}\selectfont
    \hfill {\small \textit{Written} \hspace{11.5mm} \textit{Spoken}} \hspace{3.5mm}  \hfill \\
    Finnish \hfill \scFive \mysep \scFive \\
    English \hfill \scFive \mysep \scFour \\
    Swedish \hfill \scTwo  \mysep \scOne  \\
    Spanish \hfill \scOne  \mysep \scOne  \par}
\end{tcolorbox} \boxskip

%%%%%%%% TRUST POSITIONS %%%%%%%%
\begin{tcolorbox}[top=1mm,bottom=1mm,right=1mm,left=1mm,arc=\boxarc,toptitle=\boxtop,bottomtitle=1mm,title=\sectitle{TRUST POSITIONS},box align=top]
    \titledate{Chairman}{2014 -- 2015}
    \workplace{AQ -- Aalto University Students of Acoustics} \\
    
    \titledate{Board member}{2006  -- 2008}
    \workplace{Northern Finland Street Dance Association ry}
\end{tcolorbox} \boxskip

%%%%%%%% INTERESTS %%%%%%%%
\renewcommand{\baselinestretch}{1.25} % make logos a bit bigger since the font is smaller here
\begin{tcolorbox}[top=1.25mm,bottom=2.5mm,right=1mm,left=1mm,arc=\boxarc,toptitle=\boxtop,bottomtitle=1mm,title=\sectitle{INTERESTS},box align=bottom]
    \small
    \begin{multicols}{2}
        \begin{itemize}[label={},leftmargin=0cm,labelwidth=0cm, labelsep=0cm, itemsep=0.5mm]
            \item \logopic{vinyl.pdf}   \ Music
            \item \logopic{book.pdf}    \ Sci-Fi
            \item \logopic{gym.pdf}     \ Gym
            \item \logopic{popcorn.pdf} \ Cinema
            \item \logopic{gaming.pdf}  \ PC gaming
            \item \logopic{ski.pdf}     \ Freeskiing
        \end{itemize}
    \end{multicols}
\end{tcolorbox}
\end{minipage}
\end{document}

%%%%%%%%%%%%%%%%%%%%%%%%%%%%%%%%%%%% EOF %%%%%%%%%%%%%%%%%%%%%%%%%%%%%%%%%%%%%
%%%%%%%%%%%%%%%%%%%%%%%%%%%%%%%%%%%%%%%%%%%%%%%%%%%%%%%%%%%%%%%%%%%%%%%%%%%%%%